%%% exercise 03 %%%
\documentclass{article}
\usepackage[margin=1in]{geometry}
\usepackage{amsmath, amssymb, amsfonts, enumitem, fancyvrb, fancyhdr, tikz}
% get rid of paragraph indent
\setlength{\parindent}{0 pt}
% allow section.equation numbering
\numberwithin{equation}{section}
% allows you to copy-paste code segments. requires pygments to be installed
% natively, i.e. not just in a Python venv! for example, on WSL Ubuntu you
% need to sudo apt-get install python3-pygments (unfortunately, the version
% will be a bit dated) and cannot rely on your venv Pygments version.
\usepackage{minted}
% makes clickable links to sections
\usepackage{hyperref}
% make the link colors blue, as well as cite colors. urls are magenta
\hypersetup{
    colorlinks, linkcolor = blue, citecolor = blue, urlcolor = magenta
}
% fancy pagestyle so we can use fancyhdr for fancy headers/footers
\pagestyle{fancy}
% add logo in right of header. note that you will have to adjust logo path!
\fancyhead[R]{\includegraphics[scale = 0.15]{../bac_logo1.png}}
% don't show anything in the left and center header
\fancyhead[L, C]{}
% give enough space for logo by reducing top margin height, head separator,
% increasing headerheight. see Figure 1 in the fancyhdr documentation. if
% \topmargin + \headheight + \headsep = 0, original text margins unchanged.
\setlength{\topmargin}{-60 pt}
\setlength{\headheight}{50 pt}
\setlength{\headsep}{10 pt}
% remove decorative line in the fancy header
\renewcommand{\headrulewidth}{0 pt}

% title, author + thanks, date
\title{Exercise 3}
\author{Derek Huang\thanks{NYU Stern 2021, BAC Advanced Team.}}
\date{March 10, 2021\thanks{Original version released on March 4, 2021.}}

% shortcut links. the % characters strip extra spacing.
\newcommand{\pytest}{%
    \href{https://docs.pytest.org/en/stable/}{pytest}%
}
\newcommand{\lsqr}{%
    \href{%
        https://docs.scipy.org/doc/scipy/reference/generated/%
        scipy.sparse.linalg.lsqr.html%
    }{\texttt{scipy.sparse.linalg.lsqr}}%
}
\newcommand{\skridge}{%
    \href{%
        https://scikit-learn.org/stable/modules/generated/%
        sklearn.linear_model.Ridge.html%
    }{\texttt{sklearn.linear\_model.Ridge}}%
}

\begin{document}

\maketitle
% need to include this after making title to undo the automatic
% \thispagestyle{plain} command that is issued.
\thispagestyle{fancy}

\section{Introduction}

The goal of this exercise is to implement the ridge regression model,
solving the $ \ell^2 $-regularized least-squares objective using both
matrix multiplication and \lsqr, one of the solvers used by the scikit-learn
\skridge{} ridge regression estimator.

\section{Instructions}

\subsection{General}

The \texttt{exercise\_03.py} file contains a skeleton for the
\texttt{RidgeRegression} class, two unit tests, and a \pytest{} test fixture.
Your job is to implement the \texttt{fit}, \texttt{predict}, and
\texttt{score} methods of the \texttt{RidgeRegression} class. Each one of the
three methods contains the comment block

\begin{figure}[h!]
	\centering
	\begin{BVerbatim}
###########################
### your code goes here ###
###########################
	\end{BVerbatim}
	% remove some extra spacing
	\vspace{-5 pt}
\end{figure}

Your code \textbf{must} be written in the areas marked by these blocks. Do
\textbf{not} change any of pre-written code. The exercise is complete
$ \Leftrightarrow $ \texttt{pytest /path/to/exercise\_03.py} executes with
zero test failures. Note that attributes ending with \texttt{\_}, ex.
\texttt{coef\_}, are created \textbf{during} the fitting process.

\subsection{Using \lsqr}

The \lsqr{} solver used by \skridge{} solves
\begin{equation*}
    \min_\mathbf{x}\Vert\mathbf{Ax} - \mathbf{b}\Vert_2^2 +
    d^2\Vert\mathbf{x}\Vert_2^2
\end{equation*}

Here $ \mathbf{A} \in \mathbb{R}^{N \times d} $ is the coefficient matrix,
$ \mathbf{b} \in \mathbb{R}^N $ is the coefficient vector, $ \mathbf{x} \in
\mathbb{R}^d $ is the optimization variable, and $ d \in [0, \infty) $
corresponds to the \texttt{damp} parameter passed to \lsqr.

\medskip

You should use \textbf{only} the \texttt{A}, \texttt{b}, and \texttt{damp}
parameters. See the documentation for usage examples.

\subsection{Tips}

\begin{enumerate}
    \item
    Carefully read the \texttt{RidgeRegression} method docstrings and review
    slides 10 and 12.

    \item
    The \texttt{fit} method should contain a conditional statement to handle
    when \texttt{self.solver == "matmul"} and when
    \texttt{self.solver == "lsqr"}. The unit tests will cover both cases.

    \item
    Use NumPy functions whenever possible for efficiency, brevity of code, and
    to develop good habits.

    \item
    The \textbf{square root} of \texttt{alpha} should be passed to the \lsqr{}
    \texttt{damp} parameter.
\end{enumerate}

\end{document}