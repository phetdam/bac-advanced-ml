%%% BAC exercise template %%%
% template modified by Derek Huang, original by Sean Cox.
\documentclass{article}
\usepackage[margin=1in]{geometry}
\usepackage{amsmath, amssymb, amsfonts, enumitem, fancyhdr, tikz}
% get rid of paragraph indent
\setlength{\parindent}{0 pt}
% allow section.equation numbering
\numberwithin{equation}{section}
% allows you to copy-paste code segments. requires pygments to be installed
% natively, i.e. not just in a Python venv! for example, on WSL Ubuntu you
% need to sudo apt-get install python3-pygments (unfortunately, the version
% will be a bit dated) and cannot rely on your venv Pygments version.
\usepackage{minted}
% makes clickable links to sections
\usepackage{hyperref}
% make the link colors blue, as well as cite colors. urls are magenta
\hypersetup{
    colorlinks, linkcolor = blue, citecolor = blue, urlcolor = magenta
}
% fancy pagestyle so we can use fancyhdr for fancy headers/footers
\pagestyle{fancy}
% add logo in right of header. note that you will have to adjust logo path!
\fancyhead[R]{\includegraphics[scale = 0.15]{./bac_logo1.png}}
% don't show anything in the left and center header
\fancyhead[L, C]{}
% give enough space for logo by reducing top margin height, head separator,
% increasing headerheight. see Figure 1 in the fancyhdr documentation. if
% \topmargin + \headheight + \headsep = 0, original text margins unchanged.
\setlength{\topmargin}{-60 pt}
\setlength{\headheight}{50 pt}
\setlength{\headsep}{10 pt}
% remove decorative line in the fancy header
\renewcommand{\headrulewidth}{0 pt}

% title, author + thanks, date
\title{Exercise -1}
\author{Derek Huang\thanks{NYU Stern 2021, BAC Advanced Team.}}
% the extra thanks is also useful for making notes about previous versions.
% you can remove this thanks if there are no version revisions.
\date{June 9, 2021\thanks{Original version released on February 23, 2021.}}

\begin{document}

\maketitle
% need to include this after making title to undo the automatic
% \thispagestyle{plain} command that is issued.
\thispagestyle{fancy}

% remove some space between title and first section to keep text on one page.
% comment this line out or adjust the space reduction as needed. generally,
% try to remove as little space as possible if you are trying to keep content
% on one page, else comment this line out if content spans multiple pages.
\vspace{-20 pt}

\section{PCA whitening}

Suppose we have centered data matrix $ \mathbf{X} \in
\mathbb{R}^{N \times d} $, where each of the $ N $ rows is a transposed
$ \mathbb{R}^d $ data sample. In principal components analysis we want to find
matrices $ \mathbf{W} \in \mathbb{R}^{d \times d},
\mathbf{Z} \in \mathbb{R}^{N \times d} $, such that
\begin{enumerate}
    \item
    $ \mathbf{W}^\top\mathbf{W} = \mathbf{WW}^\top = \mathbf{I} $, i.e.
    $ \mathbf{W} $ is an orthonormal matrix.

    \item
    The columns of $ \mathbf{W} $ are eigenvectors of the sample covariance
    matrix $ \mathbf{C}_\mathbf{X} \triangleq \frac{1}{N}
    \mathbf{X}^\top\mathbf{X} $ ordered in descending order by the magnitude
    of their corresponding eigenvalue.

    \item
    $ \mathbf{X} = \mathbf{ZW}^\top $, i.e. for each data point
    $ \mathbf{x}_k $, we want $ \mathbf{W}, \mathbf{Z} $ such that
    $ \mathbf{x}_k = \mathbf{Wz}_k $, $ \mathbf{z}_k $ the $ k $th row of
    $ \mathbf{Z} $.
\end{enumerate}

Suppose we find the solution by performing [compact] singular value
decomposition on $ \mathbf{X} $. That is, we find $ \mathbf{U} \in
\mathbb{R}^{N \times d} $, $ \mathbf{\Sigma} \in \mathbb{R}^{d \times d} $,
$ \mathbf{V}^\top \in \mathbb{R}^{d \times d} $ where $ \mathbf{U} $ has
orthonormal columns, $ \mathbf{\Sigma} $ is the diagonal matrix of singular
values\footnotemark\footnotetext{
    Recall that the singular values of $ \mathbf{X} $ are the square roots of
    the eigenvalues of $ N\mathbf{C}_\mathbf{X} = \mathbf{X}^\top\mathbf{X} $.
} ordered in descending order, and $ \mathbf{V} $ is an orthonormal matrix
such that $ \mathbf{X} = \mathbf{U\Sigma V}^\top $. Recall that
$ \mathbf{X} = \mathbf{ZV}^\top $, which implies that
$ \mathbf{Z} = \mathbf{XV} $, where $ \mathbf{C}_\mathbf{Z} $ is diagonal.

\medskip

What $ \tilde{\mathbf{V}} \in \mathbb{R}^{d \times d} $ can we derive from
$ \mathbf{V} $ such that $ \tilde{\mathbf{Z}} \triangleq
\mathbf{X}\tilde{\mathbf{V}}^\top $ has
$ \mathbf{C}_{\tilde{\mathbf{Z}}} = \mathbf{I} $?

\medskip

\textit{Hint.} $ \mathbf{C}_\mathbf{Z} $ is a rescaling of a particular
diagonal matrix we already computed.

\section{Flag checking}

Consider the C function using the \href{
    https://numpy.org/devdocs/reference/c-api/array.html
}{NumPy C array API} shown on the next page.

\begin{minted}{c}
static PyObject *
npy_sigmoid(PyObject *self, PyObject *args)
{
    PyArrayObject *ar;
    if (!PyArg_ParseTuple("O!", &PyArray_Type, (PyObject **) &ar)) {
        return NULL;
    }
    if (PyArray_TYPE(ar) != NPY_DOUBLE) {
        PyErr_SetString(PyExc_TypeError, "ndarray type not NPY_DOUBLE");
        return NULL;
    }
    double *data = (double *) PyArray_DATA(ar);
    npy_intp ar_size = PyArray_SIZE(ar);
    for (npy_intp i = 0; i < ar_size; i++) {
        data[i] = 1 / (1 + exp(-data[i]));
    }
    Py_RETURN_NONE;
}
\end{minted}

Why is this code dangerous? What flags must \texttt{ar} satisfy in particular
for it to be safe?

\end{document}