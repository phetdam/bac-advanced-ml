%%% exercise 08 %%%
% template modified by Derek Huang, original by Sean Cox.
\documentclass{article}
\usepackage[margin=1in]{geometry}
\usepackage{amsmath, amssymb, amsfonts, enumitem, fancyhdr, tikz}
% get rid of paragraph indent
\setlength{\parindent}{0 pt}
% allow section.equation numbering
\numberwithin{equation}{section}
% allows you to copy-paste code segments. requires pygments, which can be
% installed from PyPI with pip install Pygments.
% warning: minted does NOT work with Texmaker if you have TeX Live installed
% on WSL but Texmaker installed natively on Windows!
%\usepackage{minted}
% alternative to minted that does not require Python, LaTeX only. listings is
% however disgusting out of the box and some setup is required.
\usepackage{listings, xcolor}
% makes clickable links to sections
\usepackage{hyperref}
% make the link colors blue, as well as cite colors. urls are magenta
\hypersetup{
    colorlinks, linkcolor = blue, citecolor = blue, urlcolor = magenta
}
% fancy pagestyle so we can use fancyhdr for fancy headers/footers
\pagestyle{fancy}
% add logo in right of header. note that you will have to adjust logo path!
\fancyhead[R]{\includegraphics[scale = 0.15]{../bac_logo1.png}}
% don't show anything in the left and center header
\fancyhead[L, C]{}
% give enough space for logo by reducing top margin height, head separator,
% increasing headerheight. see Figure 1 in the fancyhdr documentation. if
% \topmargin + \headheight + \headsep = 0, original text margins unchanged.
\setlength{\topmargin}{-60 pt}
\setlength{\headheight}{50 pt}
\setlength{\headsep}{10 pt}
% remove decorative line in the fancy header
\renewcommand{\headrulewidth}{0 pt}

% color definitions for listings syntax highlighting. uses colors borrowed
% from the VS Code Dark+ and Abyss standard themes.
\definecolor{KwColor}{RGB}{153, 102, 184}     % keyword color
\definecolor{VarColor}{RGB}{86, 156, 214}     % variables/identifier color
\definecolor{StrColor}{RGB}{209, 105, 105}    % string color
\definecolor{CmtColor}{RGB}{106, 153, 85}     % comment color

% general listings configuration for all languages
\lstset{
    % change keyword, identifier, comment, string colors
    keywordstyle = \color{KwColor},
    commentstyle = \color{CmtColor},
    identifierstyle = \color{VarColor},
    stringstyle = \color{StrColor},
    % no spaces in strings
    showstringspaces = false,
    % monospace font by default
    basicstyle = \ttfamily,
    % tabsize 8 by default, this is not the 1960s
    tabsize = 4,
    % add line numbers to the left with gray typewriter font
    numbers = left,
    numberstyle = \color{gray}\ttfamily,
    % change distance from code block from 10 pt to 5 pt
    numbersep = 5 pt
}

% title, author + thanks, date
\title{Exercise 8}
\author{Derek Huang\thanks{NYU Stern 2021, BAC Advanced Team.}}
\date{June 8, 2021}

% shortcut commands. the % strips additional spacing.
\newcommand{\pytest}{\href{https://docs.pytest.org/en/stable/}{pytest}}
\newcommand{\pdb}{%
    \href{https://docs.python.org/3/library/pdb.html}{\texttt{pdb}}%
}
\newcommand{\minimize}{%
    \href{%
        https://docs.scipy.org/doc/scipy/reference/generated/%
        scipy.optimize.minimize.html%
    }{\texttt{scipy.optimize.minimize}}%
}
\newcommand{\coomatrix}{%
    \href{%
        https://docs.scipy.org/doc/scipy/reference/generated/%
        scipy.sparse.coo_matrix.html%
    }{\texttt{scipy.sparse.coo\_matrix}}%
}

\begin{document}

\maketitle
% need to include this after making title to undo the automatic
% \thispagestyle{plain} command that is issued.
\thispagestyle{fancy}

% remove some space between title and first section to keep text on one page.
% comment this line out or adjust the space reduction as needed. generally,
% try to remove as little space as possible if you are trying to keep content
% on one page, else comment this line out if content spans multiple pages.
%\vspace{-20 pt}

\section{Introduction}

The goal of this exercise is to implement a hinge loss linear support vector classifier, solving both the primal and dual formulations using \minimize{} with \texttt{method="trust-constr"}.

\section{Instructions}

\subsection{General}

The \texttt{exercise\_08.py} file contains the \texttt{LinearSVC} class, unit
tests, and \pytest{} fixtures. Complete the \texttt{fit} method of the 
\texttt{LinearSVC} class by replacing
\texttt{\#\#\# your code goes here \#\#\#} comment blocks with appropriate
Python code. Note that some comment blocks inside \texttt{if} statements may
contain the \texttt{pass} statement, which is included only to silence
language parser/linter complaints about improper block indentation.

\medskip

Your code \textbf{must} be written in the areas marked by these blocks. Do
\textbf{not} change any of the pre-written code. The exercise is complete
$ \Leftrightarrow $ \texttt{pytest /path/to/exercise\_08.py} executes with
zero test failures.

%\subsection{Using \texttt{trust-constr}}

\subsection{Primal reformulation}

Let $ \mathbf{X} \in \mathbb{R}^{N \times d} $ be the matrix of training
points, where $ \mathbf{x}_k^\top $ is the $ k $th row, and let
$ \mathbf{y} \in \{-1, 1\}^N $ be the response vector of labels. For
convenience, define the ``signed'' data matrix
$ \mathbf{Z} \in \mathbb{R}^{N \times d} $, where
\begin{equation} \label{eq:signed_X}
    \mathbf{Z} \triangleq \begin{bmatrix}
        \ y_1\mathbf{x}_1^\top \ \\ \ \vdots \ \\ \ y_N\mathbf{x}_N^\top \
    \end{bmatrix}
\end{equation}

We can then concisely write the primal linear SVC problem using
$ \mathbf{Z} $ defined in (\ref{eq:signed_X}) as
\begin{equation} \label{eq:primal_svc_opt}
    \begin{array}{ll}
        \displaystyle\min_{\mathbf{w}, b, \xi} &
            \frac{1}{2}\Vert\mathbf{w}\Vert_2^2 +
            C\mathbf{1}^\top\xi \\
        \text{s.t.} & \mathbf{Zw} + b\mathbf{y} \succeq
            \mathbf{1} - \xi  \\
            & \xi \succeq \mathbf{0}
    \end{array}
\end{equation}

Here we are jointly optimizing over $ \mathbf{w} \in \mathbb{R}^d $,
$ b \in \mathbb{R} $, $ \xi \in \mathbb{R}^N $, where $ \xi $ is the vector
of slack variables. However, we cannot directly send (\ref{eq:primal_svc_opt})
to \minimize, as it requires that the objective be a function of a single
variable, say $ \beta $. However, we can define $ \beta \in
\mathbb{R}^{d + N + 1} $ such that $ \beta^\top \triangleq
\begin{bmatrix} \ \mathbf{w}^\top & b & \xi^\top \ \end{bmatrix} $. Letting
$ f : \mathbb{R}^{d + N + 1} \rightarrow \mathbb{R} $ be the objective, a
function of $ \beta $, we thus have that $ \forall \beta \in
\mathbb{R}^{d + N + 1} $,
\begin{equation} \label{eq:primal_obj_joint}
    f(\beta) \triangleq \frac{1}{2}\beta^\top\mathbf{D}\beta +
    C\beta^\top\begin{bmatrix}
        \ \mathbf{0}_{d + 1} \ \\ \ \mathbf{1}_N \
    \end{bmatrix}
\end{equation}

Here the diagonal matrix $ \mathbf{D} \in \mathbb{R}^{(d + N + 1) \times
(d + N + 1)} $ is such that
\begin{equation}
    \mathbf{D} \triangleq \begin{bmatrix}
        \ \mathbf{I}_d & \mathbf{0}_{d \times (N + 1)} \ \\
        \ \mathbf{0}_{(N + 1) \times d} & \mathbf{0}_{(N + 1) \times (N + 1)} \
    \end{bmatrix}
\end{equation}

From (\ref{eq:primal_obj_joint}), it's clear that $ f $ is a quadratic function
of $ \beta $. Furthermore, $ \nabla f $ and $ \nabla^2 f $ are such that
\begin{equation*}
    \nabla f(\beta) \triangleq \mathbf{D}\beta + \begin{bmatrix}
        \ \mathbf{0}_{d + 1} \ \\ \ \mathbf{1}_N \
    \end{bmatrix} \quad\quad \nabla^2f(\beta) \triangleq \mathbf{D}
\end{equation*}

One can see that $ f $ is also convex, in particular that $ \mathbf{D} \succeq
\mathbf{0} $, since $ \forall \mathbf{u} \in \mathbb{R}^{d + N + 1} $, we have
\begin{equation*}
    \mathbf{u}^\top\mathbf{D}\mathbf{u} = \mathbf{u}^\top\begin{bmatrix}
        \ u_1 \ \\ \ \vdots \ \\ \ u_d \ \\ \ \mathbf{0}_{N + 1} \
    \end{bmatrix} = \sum_{j = 1}^du_j^2 \ge 0
\end{equation*}

%The linear constraint in the primal problem can be written as
%$ \begin{bmatrix} \ \mathbf{Z} & \mathbf{y} & \mathbf{I}_N \ \end{bmatrix}\beta
%\succeq \mathbf{1} $.

%Furthermore, since $ \mathbf{D} $ is very sparse, we can save memory by
%storing its nonzero elements using a sparse matrix type, for example
%\coomatrix.


%The linear SVC dual problem can be written as
%\begin{equation*}
%    \begin{array}{ll}
%        \displaystyle\min_\alpha & \frac{1}{2}\alpha^\top\mathbf{ZZ}^\top\alpha
%            - \mathbf{1}^\top\alpha \\
%        \text{s.t.} & \alpha^\top\mathbf{y} = 0 \\
%            & \mathbf{0} \preceq \alpha \preceq C\mathbf{1}
%    \end{array}
%\end{equation*}

\subsection{Tips}

\begin{enumerate}
    % need to add more important tips later

    \item
    \href{%
        https://numpy.org/doc/stable/reference/generated/numpy.hstack.html%
    }{\texttt{numpy.hstack}} is very useful for creating matrices from smaller
    blocks. For example, let $ \mathbf{A} \in \mathbb{R}^{m \times n_1} $ be
    represented by \texttt{A}, a \texttt{numpy.ndarray} with shape
    \texttt{(m, n1)}, and let $ \mathbf{B} \in \mathbb{R}^{m \times n_2} $ be
    represented by \texttt{B}, a \texttt{numpy.ndarray} with shape
    \texttt{(m, n2)}. Then, \texttt{numpy.hstack((A, B))} returns a
    \texttt{numpy.ndarray} shape \texttt{(m, n1 + n2)} representing the block
    matrix $ \begin{bmatrix} \ \mathbf{A} & \mathbf{B} \ \end{bmatrix} $.

    \item
    Use NumPy functions whenever possible for efficiency, brevity, and ease
    of debugging.

    \item
    Invoke \pytest{} with the \texttt{--pdb} flag to start \pdb, the Python
    debugger. Doing so allows you to inspect the variables in the current call
    frame and look more closely at what went wrong.
\end{enumerate}

\end{document}