%%% lecture 08 %%%
\documentclass{beamer}
\usepackage[utf8]{inputenc}
\usepackage{algorithm2e, amsmath, amssymb, amsfonts, graphicx}
% allow section.equation numbering
\numberwithin{equation}{section}
% use boadilla theme
\usetheme{Boadilla}
% remove navigation symbols
\usenavigationsymbolstemplate{}
% get numbered figure captions
\setbeamertemplate{caption}[numbered]
% changes itemize to circle + other things
\useoutertheme{split}
\useinnertheme{circles}

% command for the title string. change for each lecture
\newcommand{\lecturetitle}{Support Vector Machines}
% allow automatic alert-highlighted references and hyperlinks
\newcommand{\aref}[1]{\alert{\ref{#1}}}
\newcommand{\ahref}[2]{\href{#1}{\alert{#2}}}
% title page stuff. brackets content displayed in footer bar
\title[\lecturetitle]{\lecturetitle}
% metadata. content in brackets is displayed in footer bar
\author[Derek Huang (BAC Advanced Team)]{Derek Huang}
\institute{BAC Advanced Team}
\date{August 14, 2021}

% change "ball" bullet to numbered bullet and section title for section
\setbeamertemplate{section in toc}{\inserttocsectionnumber.~\inserttocsection}
% change ball to gray square (copied from stackoverflow; \par needed for break)
\setbeamertemplate{subsection in toc}{        
    \hspace{1.2em}{\color{gray}\rule[0.3ex]{3pt}{3pt}}~\inserttocsubsection\par
}
% use default enumeration scheme
\setbeamertemplate{enumerate items}[default]
% required line that fixes the problem of \mathbf, \bf not working in beamer
% for later (post-2019) TeX Live installations. see the issue on GitHub:
% https://github.com/josephwright/beamer/issues/630
\DeclareFontShape{OT1}{cmss}{b}{n}{<->ssub * cmss/bx/n}{}

\begin{document}

% title slide
\begin{frame}
    \titlepage
    \centering
    % relative path may need to be updated depending on .tex file location
    \includegraphics[scale = 0.1]{../bac_logo1.png}
\end{frame}

% table of contents slide
\begin{frame}{Overview}
    \tableofcontents
\end{frame}


\section{Linear SVMs}

\subsection{Maximum margin hyperplanes}

\begin{frame}{Motivation}
    \begin{itemize}
        \item
        We saw that logistic regression, LDA/QDA, and na\"{i}ve Bayes
        provide probabilistic models of [class-]conditional distributions.

        \item
        Suppose we have convex sets $ A, B \subseteq \mathbb{R}^d $. There are
        several ways to draw a distribution-free ``line'' (hyperplane) between
        them.

        \item
        Intuitively, the optimal hyperplane is furthest away from both
        $ A, B $. How can solve the resulting optimization problem?

        \item
        How do we define ``furthest away'' in this context?
    \end{itemize}
\end{frame}

%\begin{frame}{Maximum margin hyperplanes}
%    \begin{itemize}
%        \item
%        
%    \end{itemize}
%\end{frame}
%
%\subsection{Primal formulation}
%
%\begin{frame}{Primal formulation}
%    \begin{itemize}
%        \item
%    \end{itemize}
%\end{frame}
%
%\subsection{Dual formulation}

% BibTeX slide for references. should use either acm or ieeetr style
\begin{frame}{References}
    \bibliographystyle{acm}
    % relative path may need to be updated depending on .tex file location
    \bibliography{../master_bib}
\end{frame}

\end{document}