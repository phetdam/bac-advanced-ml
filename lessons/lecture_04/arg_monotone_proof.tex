%%% BAC exercise template %%%
% template modified by Derek Huang, original by Sean Cox.
\documentclass{article}
\usepackage[margin=1in]{geometry}
\usepackage{amsmath, amssymb, amsfonts, enumitem, fancyhdr, tikz}
% get rid of paragraph indent
\setlength{\parindent}{0 pt}
% allow section.equation numbering
\numberwithin{equation}{section}
% allows you to copy-paste code segments. requires pygments, which can be
% installed from PyPI with pip install Pygments.
% warning: minted does NOT work with Texmaker if you have TeX Live installed
% on WSL but Texmaker installed natively on Windows!
%\usepackage{minted}
% alternative to minted that does not require Python, LaTeX only. listings is
% however disgusting out of the box and some setup is required.
\usepackage{listings, xcolor}
% makes clickable links to sections
\usepackage{hyperref}
% make the link colors blue, as well as cite colors. urls are magenta
\hypersetup{
    colorlinks, linkcolor = blue, citecolor = blue, urlcolor = magenta
}
% fancy pagestyle so we can use fancyhdr for fancy headers/footers
\pagestyle{fancy}
% add logo in right of header. note that you will have to adjust logo path!
\fancyhead[R]{\includegraphics[scale = 0.15]{../bac_logo1.png}}
% don't show anything in the left and center header
\fancyhead[L, C]{}
% give enough space for logo by reducing top margin height, head separator,
% increasing headerheight. see Figure 1 in the fancyhdr documentation. if
% \topmargin + \headheight + \headsep = 0, original text margins unchanged.
\setlength{\topmargin}{-60 pt}
\setlength{\headheight}{50 pt}
\setlength{\headsep}{10 pt}
% remove decorative line in the fancy header
\renewcommand{\headrulewidth}{0 pt}

% color definitions for listings syntax highlighting. uses colors borrowed
% from the VS Code Dark+ and Abyss standard themes.
\definecolor{KwColor}{RGB}{153, 102, 184}     % keyword color
\definecolor{VarColor}{RGB}{86, 156, 214}     % variables/identifier color
\definecolor{StrColor}{RGB}{209, 105, 105}    % string color
\definecolor{CmtColor}{RGB}{106, 153, 85}     % comment color

% general listings configuration for all languages
\lstset{
    % change keyword, identifier, comment, string colors
    keywordstyle = \color{KwColor},
    commentstyle = \color{CmtColor},
    identifierstyle = \color{VarColor},
    stringstyle = \color{StrColor},
    % no spaces in strings
    showstringspaces = false,
    % monospace font by default
    basicstyle = \ttfamily,
    % tabsize 8 by default, this is not the 1960s
    tabsize = 4,
    % add line numbers to the left with gray typewriter font
    numbers = left,
    numberstyle = \color{gray}\ttfamily,
    % change distance from code block from 10 pt to 5 pt
    numbersep = 5 pt
}

% title, author + thanks, date
\title{Useful $ \arg\max $, $ \arg\min $ properties}
\author{Derek Huang\thanks{NYU Stern 2021, BAC Advanced Team.}}
% the extra thanks is also useful for making notes about previous versions.
% you can remove this thanks if there are no version revisions.
\date{March 14, 2021}

\begin{document}

\maketitle
% need to include this after making title to undo the automatic
% \thispagestyle{plain} command that is issued.
\thispagestyle{fancy}

\section{Introduction}

The $ \arg\max $, $ \arg\min $ operators are widely used in the context of
unconstrained optimization and have several useful properties that allow one
to replace an objective $ f $ with a simpler objective $ g $ under certain
conditions. Here we list and prove several of these properties to demonstrate
why they are true.

\section{Properties}

Let $ f : \mathbb{R}^n \rightarrow \mathbb{R} $ be a function that attains its
maximum at $ \mathbf{x}_\uparrow $ and its minimum at
$ \mathbf{x}_\downarrow $\footnote{
    The properties still hold if $ f $ attains its maximum at any point in
    nonempty $ X_\uparrow \subseteq \mathbb{R}^n $ and its maximum at any point
    in nonempty $ X_\downarrow \subseteq \mathbb{R}^n $. For example, consider
    $ f(\mathbf{x}) \triangleq c $, $ c \in \mathbb{R} $. Here $ X_\uparrow =
    X_\downarrow = \mathbb{R}^n $.
}.

\medskip

\textit{Proposition.} For $ g : \mathbb{R} \rightarrow \mathbb{R} $,
$ g(y) = ay + b $, $ a \in (0, \infty), b \in \mathbb{R} $,
\begin{equation*}
    \begin{split}
        \arg\max_\mathbf{x}f(\mathbf{x}) & =
        \arg\max_\mathbf{x}g(f(\mathbf{x})) \\
        \arg\min_\mathbf{x}f(\mathbf{x}) & =
        \arg\min_\mathbf{x}g(f(\mathbf{x}))
    \end{split}
\end{equation*}

\medskip

\textit{Proof.} Let $ \tilde{\mathbf{x}} =
\arg\max_\mathbf{x}g(f(\mathbf{x})) $. By definition, $ \forall \mathbf{x}
\in \mathbb{R}^n $,
\begin{equation*}
    g(\tilde{\mathbf{x}}) \ge g(\mathbf{x}) \Rightarrow
    af(\tilde{\mathbf{x}}) + b \ge af(\mathbf{x}) + b \Leftrightarrow
    f(\tilde{\mathbf{x}}) \ge f(\mathbf{x}) \Rightarrow
    f(\tilde{\mathbf{x}}) \ge f(\mathbf{x}_\uparrow)
\end{equation*}

By definition, $ \forall \mathbf{x} \in \mathbb{R}^n $,
$ f(\mathbf{x}_\uparrow) \ge f(\mathbf{x}) \Rightarrow
f(\mathbf{x}_\uparrow) \ge f(\tilde{\mathbf{x}}) $. $ f(\mathbf{x}_\uparrow)
\ge f(\tilde{\mathbf{x}}) $, $ f(\tilde{\mathbf{x}}) \ge
f(\mathbf{x}_\uparrow) \Rightarrow
f(\mathbf{x}_\uparrow) = f(\tilde{\mathbf{x}}) $. $ f $ attains a unique
maximum at $ \hat{\mathbf{x}} \Rightarrow
\mathbf{x}_\uparrow = \tilde{\mathbf{x}} $. Therefore,
$\arg\max_\mathbf{x}f(\mathbf{x}) = \mathbf{x}_\uparrow = \tilde{\mathbf{x}} =
\arg\max_\mathbf{x}g(f(\mathbf{x})) $.

\bigskip

\begin{enumerate}
    \item \label{ppp}
    \textit{Proposition.} \ref{ppp}
\end{enumerate}


\end{document}