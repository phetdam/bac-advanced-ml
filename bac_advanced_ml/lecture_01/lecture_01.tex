\documentclass{beamer}
\usepackage[utf8]{inputenc}
\usepackage{amsmath, amssymb, amsfonts, graphicx}
% use boadilla theme
\usetheme{Boadilla}
% remove navigation symbols
\usenavigationsymbolstemplate{}
% get numbered figure captions
\setbeamertemplate{caption}[numbered]
% changes itemize to circle + other things
\useoutertheme{split}
\useinnertheme{circles}

% title page stuff
\title[Math for Data Science]{Math for Data Science}
\author[Derek Huang (BAC Advanced Team)]{Derek Huang}
\institute[BAC Advanced Team]{BAC Advanced Team}
\date[December 26, 2020]{December 26, 2020}

% change "ball" bullet to numbered bullet and section title for section
\setbeamertemplate{section in toc}{\inserttocsectionnumber.~\inserttocsection}
% change ball to gray square (copied from stackoverflow; \par needed for break)
\setbeamertemplate{subsection in toc}{        
    \hspace{1.2em}{\color{gray}\rule[0.3ex]{3pt}{3pt}}~\inserttocsubsection\par}
% use default enumeration scheme
\setbeamertemplate{enumerate items}[default]
% required line that fixes the problem of \mathbf, \bf not working in beamer
% for later (post-2019) TeX Live installations. see the issue on GitHub:
% https://github.com/josephwright/beamer/issues/630
\DeclareFontShape{OT1}{cmss}{b}{n}{<->ssub * cmss/bx/n}{}

\begin{document}

% title slide
\begin{frame}
    \titlepage
    \centering
    \includegraphics[scale = 0.1]{../bac_logo1.png}
\end{frame}

% table of contents
\begin{frame}{Overview}
	\tableofcontents
\end{frame}

\section{Introduction}

\subsection{Motivation}

\begin{frame}{Motivation}
    \begin{itemize}
        \item
        Why do we need a math review/primer?
        
        \item
        Short answer: so that we can be \textit{informed} model users.
        
        \item
        Being an informed user is important! It's not so simple as just taking
        data and chucking it into a model--doing exactly that can give
        \textit{seriously misleading} modeling results.

        \item
        \textit{Example}. Principal components analysis (PCA). A typical
        mistake people make\footnotemark\footnotetext{
            I did this before but caught my mistake before I turned in my
            analysis.
        } is to perform PCA on data that has not been scaled to have unit
        per-component variances.

        \item
        Since PCA is scaling sensitive, one often ends up with a couple massive
        and many small eigenvalues/singular values \footnotemark\footnotetext{
            Define $ \mathbf{X} \in \mathbb{R}^{N \times d} $ as the data
            matrix with sample covariance matrix $ \mathbf{C} \triangleq
            \frac{1}{N}\mathbf{X}^\top\mathbf{X} $. Eigendecomposition is used
            on $ \mathbf{C} $ while singular value decomposition is used on
            $ \mathbf{X} $.
        }. This may result in the incorrect conclusion that the data may be
        well approximated by a lower-dimensional orthogonal eigenbasis.

        % provide some extra space before footnotes
        \medskip
    \end{itemize}
\end{frame}

\subsection{Topics}

\begin{frame}{Topics}
    \begin{itemize}
        \item
        It can only help and never hurt to know more
        math\footnotemark\footnotetext{
            Interestingly, \href{
                https://cseweb.ucsd.edu/~yfreund/papers/brownboost.pdf
            }{\alert{Freund's BrownBoost}} algorithm has connections to
            Brownian motion and requires the solution to a differential
            equation at each iteration.
        }, but our time is
        limited. Therefore, we will only briefly cover enough mathematical
        knowledge to give a useful foundation for self-learning.

        \item
        In general, knowing all the following topics at an undergraduate or
        early graduate level is ideal: probability theory, linear algebra,
        vector/matrix calculus, and some convex/nonlinear optimization.

        \item
        It goes without mentioning that an elementary understanding of set
        theory is necessary. Knowing some real analysis is helpful.
    \end{itemize}
\end{frame}

\section{Set theory}

\begin{frame}{Set theory}
    \begin{itemize}
        \item
        \textit{Definition.} A \textit{set} is a collection of non-duplicate
        items.

        \item
        \textit{Examples.}
        \begin{itemize}
            \item
            \textit{Sample space of coin flip.} H for heads, T for tails. The
            sample space $ \Omega \triangleq \{\text{H}, \text{T}\} $, where
            $ \triangleq $ means ``is defined as''.

            \item
            \textit{Natural numbers.} A special set, denoted with
            $ \mathbb{N} $. The set of numbers $ 1, 2, $ etc. This is a
            \textit{countably infinite} set, as it has an infinite number of
            elements, but the elements can be ``counted out''.

            \item
            \textit{Real numbers.} A special set, denoted with $ \mathbb{R} $.
            This type of set if \textit{uncountably infinite}, as you can't
            ``count out'' all the real numbers.

            \item
            \textit{Open Euclidean ball}. $ B(\mathbf{x}, r) \triangleq
            \{\mathbf{x}' \in \mathbb{R}^n : \Vert\mathbf{x}'\Vert_2 < r\} $,
            where $ \in $ means ``in''. The set of points in $ \mathbb{R}^n $
            centered around $ \mathbf{x} $ whose 2-norm is strictly less than
            $ r \in (0, \infty) $. A \textit{closed ball} replaces $ < $ with
            $ \le $.
        \end{itemize}

        \item
        \textit{Definition.} The \textit{cardinality} of a set $ A $, denoted
        by $ |A| $, gives the number of elements in the set. Cardinality is
        mostly used in the context of finite sets---infinite sets are a
        different story.
    \end{itemize}
\end{frame}

\begin{frame}{Set theory}
    \begin{itemize}
        \item
        \textit{Definition.} A \textit{subset} $ B $ of a set $ A $ is a set
        containing only some elements from $ A $. Usually $ B \subseteq A $
        means that $ B $ might be equal to $ A $ while $ B \subset A $ means
        that $ B $ is a \textit{strict} subset of $ A $.

        \item
        \textit{Examples.}
        \begin{itemize}
            \item
            \textit{Empty set.} A set containing nothing. For any set $ A $,
            $ \emptyset \subseteq A $.

            \item
            \textit{Power set.} Denote $ \Omega \triangleq \{\text{H},
            \text{T}\} $ as the coin flip sample space. The power set
            associated with $ \Omega $, the set of all possible subsets of
            $ \Omega $, is denoted with $ 2^\Omega \triangleq \{\emptyset,
            \{\text{H}\}, \{\text{T}\}, \{\text{H}, \text{T}\}\} $.
        \end{itemize}

        \item
        \textit{Definition.} For sets $ A, B $, $ A \cap B $ denotes the 
        \textit{intersection} of $ A, B $, i.e. the set that contains only
        elements common to $ A, B $.

        \item
        \textit{Definition.} For sets $ A, B $, $ A \cup B $ denotes the
        \textit{union} of $ A, B $, i.e. the set containing all the elements
        of $ A, B $.
    \end{itemize}
\end{frame}

\begin{frame}{Set theory}
    \begin{itemize}
        \item
        \textit{Definition.} Two sets $ A, B $ are \textit{disjoint} if
        $ A \cap B = \emptyset $.

        \item
        \textit{Definition.} Sets $ A_1, A_2 \ldots $ are
        \textit{mutually disjoint} if $ \bigcap_{k = 1}^\infty A_k =
        \emptyset $.

        \item
        \textit{Definition.} For sets $ A, B $, the \textit{set difference}
        $ A \setminus B $ is the set containing all elements in $ A $ that are
        not in $ B $.

        \item
        \textit{Examples.}
        \begin{itemize}
            \item
            $ \mathbb{R} \setminus [0, \infty) = (-\infty, 0) $, the set of
            negative real numbers. 

            \item
            $ [0, \infty) \setminus \mathbb{R} = \emptyset $. Set difference
            is \textit{not} a symmetric relation.

            \item
            Denote $ \mathbb{Q} $ as the set of rationals.
            $ \mathbb{R} \setminus \mathbb{Q} $ is the set of irrationals.
        \end{itemize}

        \item
        \textit{Definition.} Denote $ \Omega $ as the \textit{universal set}
        and define $ A \subseteq \Omega $. The \textit{complement} of $ A $
        with respect to $ \Omega $ is $ A^\mathsf{c} \triangleq \Omega
        \setminus A $.

        \item
        \textit{Definition.} The \textit{Cartesian product} of sets
        $ X_1, \ldots X_n $, $ \mathcal{C} \triangleq
        X_1 \times \ldots X_n $, is the set of all $ n $-tuples whose
        $ i $th element is an element of $ X_i $.
    \end{itemize}
\end{frame}

\begin{frame}{Set theory}
    \begin{itemize}
        \item
        \textit{Definition.} A  $ \sigma $\textit{-algebra}, or
        $ \sigma $\textit{-field}, on a set $ \Omega $ is a collection
        $ \mathcal{F} $ of subsets of $ \Omega $ that has the following
        properties \cite{tamuz_prob}:
        \begin{enumerate}
            \item
            $ \Omega \in \mathcal{F} $.

            \item
            \textit{Closure under complement.} $ A \in \mathcal{F} \Rightarrow
            A^\mathsf{c} \in \mathcal{F} $.
            \begin{itemize}
                \item
                Note that since $ \Omega \in \mathcal{F} $, then clearly
                $ \emptyset \in \mathcal{F} $.
            \end{itemize}

            \item
            \textit{Closure under countable unions.} $ A_1, A_2, \ldots
            \mathcal{F} \Rightarrow \bigcup_{k = 1}^\infty
            A_k \in \mathcal{F} $.

            \item
            \textit{Closure under countable intersections.} $ A_1, A_2,
            \ldots \mathcal{F} \Rightarrow \bigcap_{k = 1}^\infty A_k
            \mathcal{F} $.
        \end{enumerate}

        \item
        \textit{Definition.} A \textit{function} $ f : X \rightarrow Y $ is a
        mapping from a set $ X $ to a set $ Y $. We may also write
        $ f \triangleq \{(x, y) \in X \times Y : y = f(x)\} $, calling $ f $
        the set of ordered pairs $ (x, y) \in X \times Y $ where $ y = f(x) $.

        \item
        \textit{Definition.} The \textit{domain} of $ f $ is $ X $, also
        denoted as $ \operatorname{dom} f $, while the \textit{codomain} of
        $ f $ is $ Y $. The \textit{image} or \textit{range} of $ f $, denoted
        as $ \operatorname{im} f $ or $ f(X) $, is defined such that
        $ \operatorname{im} f \triangleq f(X) = \{f(x) \in Y : x \in X\} $.
    \end{itemize}
\end{frame}

\section{Probability}

\subsection{Preliminaries}

\begin{frame}{Preliminaries}
    \begin{itemize}
        \item
        \textit{Definition.} A \textit{sample space} is the universal set
        $ \Omega $ of outcomes.

        \item
        \textit{Definition.} A set $ A \subseteq \Omega $, $ \Omega $ a sample
        space, is an \textit{event}.

        \item
        \textit{Definition.} A \textit{probability measure} $ \mathbb{P} $
        defined on a sample space $ \Omega $ is a function from a
        $ \sigma $\textit{-algebra} $ \mathcal{F} \subseteq 2^\Omega $ to the
        unit interval.

        \item
        \textit{Definition.} A \textit{probability space} is the 3-tuple
        $ (\Omega, \mathcal{F}, \mathbb{P}) $. $ \Omega $ is the universal set,
        $ \mathcal{F} $ is a $ \sigma $-algebra, and $ \mathbb{P} $ is a
        probability measure.

        \item
        \textit{Examples.}
        \begin{itemize}
            \item
            \textit{Probability space of a fair coin.}
            $ (\Omega, \mathcal{F}, \mathbb{P}) $, where $ \Omega \triangleq
            \{\text{H}, \text{T}\} $, $ \mathcal{F} \triangleq \{\emptyset,
            \{\text{H}\}, \{\text{T}\}, \Omega\} = 2^\Omega $.
            $ \mathbb{P}(\{\text{H}\}) = \mathbb{P}(\{T\}) = 1 / 2 $ since the
            coin is fair. Note that $ \mathbb{P}(\emptyset) = 0 $,
            $ \mathbb{P}(\Omega) = 1 $. Why?

            \item
            \textit{Probability space of 3-sided die.} $ (\Omega, \mathcal{F},
            \mathbb{P}) $, where $ \Omega \triangleq \{1, 2, 3\} $,
            $ \mathcal{F} \triangleq \{\emptyset, \{1\}, \{2\}, \{3\},
            \{1, 2\}, \{1, 3\}, \{2, 3\}, \Omega\} $. If $ A \triangleq $
            ``roll an odd number'', $ A = \{1, 3\} $ and
            $ \mathbb{P}(A) = \mathbb{P}(\{1\}) + \mathbb{P}(\{3\}) = 2 / 3 $.
            Makes intuitive sense, but why is this formally true?
        \end{itemize}
    \end{itemize}
\end{frame}

\begin{frame}{Preliminaries}
    \begin{itemize}
        \item
        \textit{Kolmogorov axioms} \cite{all_of_stats}.
        \begin{enumerate}
            \item
            $ \forall A \in \mathcal{F} $, $ \mathbb{P}(A) \ge 0 $, i.e. any
            event has a nonnegative probability.

            \item
            $ \mathbb{P}(\Omega) = 1 $, i.e. the probability that anything
            happens is 1.

            \item
            For countably infinite, disjoint $ A_1, A_2, \ldots $, then
            \begin{equation*}
                \mathbb{P}\left(\bigcup_{k = 1}^\infty A_k\right) =
                \sum_{k = 1}^\infty\mathbb{P}(A_k)
            \end{equation*}
            The probability of the union of mutually disjoint events is equal
            to the sum of the individual events' probabilities.
        \end{enumerate}

        \item
        \textit{Consequences of the axioms.}
        \begin{enumerate}
            \item
            $ \mathbb{P}(\emptyset) = 1 - \mathbb{P}(\Omega) = 0 $.

            \item
            $ \forall A, B \subseteq \Omega $, $ \mathbb{P}(A \cup B) =
            \mathbb{P}(A) + \mathbb{P}(B) - \mathbb{P}(A \cap B) $. $ \forall $
            means ``for all''.
        \end{enumerate}
    \end{itemize}
\end{frame}

\begin{frame}{Preliminaries}
    \begin{itemize}
        \item
        \textit{Definition.} Events $ A, B \subseteq \Omega $
        \textit{independent} $ \Leftrightarrow \mathbb{P}(A \cap B) =
        \mathbb{P}(A)\mathbb{P}(B) $. A common notation for independent
        $ A, B $ is $ A \perp B $.

        \item
        \textit{Remark.} Mutual exclusion \textbf{does not} imply independence,
        and vice versa. Suppose $ \Omega = \{\text{H}, \text{T}\} $. Note that
        $ \{\text{H}\} \cap \{\text{T}\} = \emptyset $, as the coin is not a
        quantum coin, and that $ \mathbb{P}(\emptyset) = 0 \ne 1 / 4 =
        \mathbb{P}(\{\text{H}\})\mathbb{P}(\{\text{T}\}) $.

        \item
        \textit{Definition.} For $ A, B \subseteq \Omega $,
        $ \mathbb{P}(B) \ne 0 $, define
        \begin{equation*}
            \mathbb{P}(A \mid B) \triangleq \frac{\mathbb{P}(A \cap B)}{
                \mathbb{P}(B)            
            }
        \end{equation*}
        $ \mathbb{P}(A \mid B) $ is the \textit{conditional probability} of
        $ A $ given $ B $.

        \item
        \textit{Remark.} Conditional probabilities are \textbf{not}
        well-defined when $ \mathbb{P}(B) = 0 $. Suppose $ B = \emptyset $.
        Then, $ \mathbb{P}(B) = 0 $, and $ \forall A \in \mathcal{F} $,
        $ \mathbb{P}(A \cap B) = 0 $. Intuitively
        $ \mathbb{P}(A \mid B) = 0 $, but the definition breaks.
    \end{itemize}
\end{frame}

\subsection{Random variables}

\begin{frame}{Random variables}
    \begin{itemize}
        \item
        \textit{Definition.} A \textit{random variable} is a function
        $ X : \Omega \rightarrow \mathbb{F} $, $ \mathbb{F} $ a
        set. Usually $ \mathbb{F} $ is either $ \mathbb{R} $,
        $ \mathbb{R}^n $, or a subset of either.
    \end{itemize}
\end{frame}

\section{Linear algebra}

\begin{frame}{References}
    \begin{thebibliography}{99}
        \bibitem{all_of_stats}
        Wasserman, L. (2004). \textit{All of Statistics: A Concise Course in
        Statistical Inference}. Springer Science \& Business Media.

        \bibitem{algebra}
        Judson, T. W. (2020). \textit{Abstract Algebra: Theory and
        Applications}. \url{http://abstract.ups.edu/download.html}.

        \bibitem{tamuz_prob}
        Tamuz, Omer. (2018). \textit{Lecture Notes on Probability}. Personal
        collection of O. Tamuz, California Institute of Technology, Pasadena,
        California.
        \url{http://tamuz.caltech.edu/teaching/ma144a/lectures.pdf}/        
    \end{thebibliography}
\end{frame}

\end{document}
