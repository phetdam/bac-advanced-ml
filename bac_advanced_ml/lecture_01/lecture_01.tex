\documentclass{beamer}
\usepackage[utf8]{inputenc}
\usepackage{amsmath, amssymb, amsfonts, graphicx}
% use boadilla theme
\usetheme{Boadilla}
% remove navigation symbols
\usenavigationsymbolstemplate{}
% get numbered figure captions
\setbeamertemplate{caption}[numbered]
% changes itemize to circle + other things
\useoutertheme{split}
\useinnertheme{circles}

% title page stuff
\title[Math for Data Science]{Math for Data Science}
\author[Derek Huang (BAC Advanced Team)]{Derek Huang}
\institute[BAC Advanced Team]{BAC Advanced Team}
\date[December 26, 2020]{December 26, 2020}

% change "ball" bullet to numbered bullet and section title for section
\setbeamertemplate{section in toc}{\inserttocsectionnumber.~\inserttocsection}
% change ball to gray square (copied from stackoverflow; \par needed for break)
\setbeamertemplate{subsection in toc}{        
    \hspace{1.2em}{\color{gray}\rule[0.3ex]{3pt}{3pt}}~\inserttocsubsection\par}
% use default enumeration scheme
\setbeamertemplate{enumerate items}[default]
% required line that fixes the problem of \mathbf, \bf not working in beamer
% for later (post-2019) TeX Live installations. see the issue on GitHub:
% https://github.com/josephwright/beamer/issues/630
\DeclareFontShape{OT1}{cmss}{b}{n}{<->ssub * cmss/bx/n}{}

\begin{document}

% title slide
\begin{frame}
    \titlepage
    \centering
    \includegraphics[scale = 0.1]{../bac_logo1.png}
\end{frame}

% table of contents
\begin{frame}{Overview}
	\tableofcontents
\end{frame}

\section{Introduction}

\subsection{Motivation}

\begin{frame}{Motivation}
    \begin{itemize}
        \item
        Why do we need a math review/primer?
        
        \item
        Short answer: so that we can be \textit{informed} model users.
        
        \item
        Being an informed user is important! It's not so simple as just taking
        data and chucking it into a model--doing exactly that can give
        \textit{seriously misleading} modeling results.

        \item
        \textit{Example}. Principal components analysis (PCA). A typical
        mistake people make\footnotemark\footnotetext{
            I did this before but caught my mistake before I turned in my
            analysis.
        } is to perform PCA on data that has not been scaled to have unit
        per-component variances.

        \item
        Since PCA is scaling sensitive, one often ends up with a couple massive
        and many small eigenvalues/singular values \footnotemark\footnotetext{
            Define $ \mathbf{X} \in \mathbb{R}^{N \times d} $ as the data
            matrix with sample covariance matrix $ \mathbf{C} \triangleq
            \frac{1}{N}\mathbf{X}^\top\mathbf{X} $. Eigendecomposition is used
            on $ \mathbf{C} $ while singular value decomposition is used on
            $ \mathbf{X} $.
        }. This may result in the incorrect conclusion that the data may be
        well approximated by a lower-dimensional orthogonal eigenbasis.

        % provide some extra space before footnotes
        \medskip
    \end{itemize}
\end{frame}

\subsection{Topics}

\begin{frame}{Topics}
    \begin{itemize}
        \item
        It can only help and never hurt to know more
        math\footnotemark\footnotetext{
            Interestingly, \href{
                https://cseweb.ucsd.edu/~yfreund/papers/brownboost.pdf
            }{\alert{Freund's BrownBoost}} algorithm has connections to
            Brownian motion and requires the solution to a differential
            equation at each iteration.
        }, but our time is
        limited. Therefore, we will only briefly cover enough mathematical
        knowledge to give a useful foundation for self-learning.

        \item
        In general, knowing all the following topics at an undergraduate or
        early graduate level is ideal: probability theory, linear algebra,
        vector/matrix calculus, and some convex/nonlinear optimization.

        \item
        It goes without mentioning that an elementary understanding of set
        theory is necessary. Knowing some real analysis can be helpful.
    \end{itemize}
\end{frame}

\section{Discrete math}

\subsection{Set theory}

\begin{frame}{Set theory}
    \begin{itemize}
        \item
        \textit{Definition.} A \textit{set} is a collection of non-duplicate
        items.

        \item
        \textit{Examples.}
        \begin{itemize}
            \item
            \textit{Sample space of coin flip.} H for heads, T for tails. The
            sample space $ \Omega \triangleq \{\text{H}, \text{T}\} $, where
            $ \triangleq $ means ``is defined as''.

            \item
            \textit{Natural numbers.} A special set, denoted with
            $ \mathbb{N} $. Basically the set containing numbers
            $ 1, 2, $ etc. This is a \textit{countably infinite} set, as it
            has an infinite number of elements but they can be ``counted out''.

            \item
            \textit{Real numbers.} A special set, denoted with $ \mathbb{R} $.
            This type of set if \textit{uncountably infinite}, as you can't
            ``count out'' all the real numbers.

            \item
            \textit{Open Euclidean ball}. $ B(\mathbf{x}, r) \triangleq
            \{\mathbf{x}' \in \mathbb{R}^n : \Vert\mathbf{x}'\Vert_2 < r\} $.
            The set of points in $ \mathbb{R}^n $ centered around
            $ \mathbf{x} $ whose 2-norm is strictly less than
            $ r \in (0, \infty) $. A \textit{closed ball} replaces $ < $ with
            $ \le $.
        \end{itemize}

        \item
        \textit{Definition.} The \textit{cardinality} of a set $ A $, denoted
        by $ |A| $, gives the number of elements in the set. Cardinality is
        mostly used in the context of finite sets---infinite sets are a
        different story.
    \end{itemize}
\end{frame}

\begin{frame}{Set theory}
    \begin{itemize}
        \item
        \textit{Definition.} A \textit{subset} $ B $ of a set $ A $ is a set
        containing only some elements from $ A $. Usually $ B \subseteq A $
        means that $ B $ might be equal to $ A $ while $ B \subset A $ means
        that $ B $ is a \textit{strict} subset of $ A $.

        \item
        \textit{Examples.}
        \begin{itemize}
            \item
            \textit{Empty set.} A set containing nothing. For any set $ A $,
            $ \emptyset \subseteq A $.

            \item
            \textit{Power set.} Denote $ \Omega \triangleq \{\text{H},
            \text{T}\} $ as the coin flip sample space. The power set
            associated with $ \Omega $, the set of all possible subsets of
            $ \Omega $, is denoted with $ 2^\Omega \triangleq \{\emptyset,
            \{\text{H}\}, \{\text{T}\}, \{\text{H}, \text{T}\}\} $.
            \begin{itemize}
                \item
                Note that $ \mathbb{P}(\{\text{H}, \text{T}\}) = 0 $, as the
                coin is not a quantum coin, and that $ \mathbb{P}(\emptyset)
                = 0 $ as clearly \textit{something} must happen.
            \end{itemize}
        \end{itemize}

        \item
        \textit{Definition.} For sets $ A, B $, $ A \cap B $ denotes the 
        \textit{intersection} of $ A, B $, i.e. the set that contains only
        elements common to $ A, B $.

        \item
        \textit{Definition.} For sets $ A, B $, $ A \cup B $ denotes the
        \textit{union} of $ A, B $, i.e. the set containing all the elements
        of $ A, B $.
    \end{itemize}
\end{frame}

\subsection{Functions}

\begin{frame}{Functions}
    \begin{itemize}
        \item
        \textit{Definition.} The \textit{Cartesian product} $ \mathcal{C} $ of
        sets $ X_1, \ldots X_n $ is denoted as $ \mathcal{C} \triangleq
        X_1 \times \ldots X_n $ and is the set of all $ n $-tuples whose
        $ i $th element is an element of $ X_i $. A common example is
        $\mathbb{R}^n $, $ n $-dimensional Euclidean space, special notation
        for $ \mathbb{R} \times \ldots \mathbb{R} $.

        \item
        \textit{Definition.} A \textit{function} $ f : X \rightarrow Y $ is a
        mapping from a set $ X $ to a set $ Y $. We may also write
        $ f \triangleq \{(x, y) \in X \times Y : y = f(x)\} $, calling $ f $
        the set of ordered pairs $ (x, y) \in X \times Y $ where $ y = f(x) $.

        \item
        \textit{Definition.} The \textit{domain} of $ f $ is $ X $, also
        denoted as $ \operatorname{dom} f $, while the \textit{codomain} of
        $ f $ is $ Y $. The \textit{image} or \textit{range} of $ f $, denoted
        as $ \operatorname{im} f $, is defined such that
        $ \operatorname{im} f = \{f(x) \in Y : x \in X\} $.
    \end{itemize}
\end{frame}

\begin{frame}{Functions}
    \begin{itemize}
        \item
        \textit{Definition.} A function $ f : X \rightarrow Y $ is
        \textit{injective}, or \textit{one-to-one}, if for $ x_1, x_2 \in X $,
        $ x_1 \ne x_2 \Rightarrow f(x_1) \ne f(x_2) $. $ \Rightarrow $ means
        ``implies''.

        \item
        \textit{Examples.}
        \begin{itemize}
            \item
            $ f : \mathbb{Z} \rightarrow \mathbb{Z} $
        \end{itemize}
    \end{itemize}
\end{frame}

\section{Probability}

\subsection{Axioms}

\begin{frame}{Axioms}
    \begin{itemize}
        \item
        \textit{Definition.} A \textit{probability measure} $ \mathbb{P} $
        defined on a sample space $ \Omega $ is a function from a
        \textit{sigma-algebra} $ \mathcal{F} \subseteq 2^\Omega $ to the unit
        interval.

        \item
        \textit{Kolmogorov axioms} \cite{all_of_stats}.
        \begin{enumerate}
            \item
            $ \forall A \in \mathcal{F} $, $ \mathbb{P}(A) \ge 0 $, i.e. any
            event has a nonnegative probability.

            \item
            $ \mathbb{P}(\Omega) = 1 $, i.e. the probability that anything
            happens is 1.

            \item
            For countable infinite, disjoint $ A_1, A_2, \ldots $, then
            \begin{equation*}
                \mathbb{P}\left(\bigcup_{k = 1}^\infty A_k\right) =
                \sum_{k = 1}^\infty\mathbb{P}(A_k)
            \end{equation*}
            The probability of the union of mutually disjoint events is equal
            to the sum of the individual events' probabilities.
        \end{enumerate}
    \end{itemize}
\end{frame}

\begin{frame}{Axioms}
    \begin{itemize}
        \item
        \textit{Example. Proving} $ \mathbb{P}(A \cup B) = \mathbb{P}(A) +
        \mathbb{P}(B) - \mathbb{P}(A \cap B) $.

        \item
        \textit{Proof.} Write $ A, B, A \cup B $ as unions of mutually disjoint
        events. $ A = (A \cap B) \cup (A \cap B^\mathsf{c}) $,
        $ B = (A \cap B) \cup (A^\mathsf{c} \cap B) $, $ A \cup B = (A \cap B)
        \cup (A \cap B^\mathsf{c}) \cup (A^\mathsf{c} \cap B) $. By Axiom 3,
        \begin{equation*}
            \begin{split}
                \mathbb{P}(A \cup B) & = \mathbb{P}(A \cap B) +
                \mathbb{P}(A \cap B^\mathsf{c}) +
                \mathbb{P}(A^\mathsf{c} \cap B) \\
                & = \mathbb{P}(A \cap B) + \mathbb{P}(A \cap B^\mathsf{c}) +
                \mathbb{P}(A \cap B) \\
                & \quad + \mathbb{P}(A^\mathsf{c} \cap B) -
                \mathbb{P}(A \cap B) \\
                & = \mathbb{P}(A) + \mathbb{P}(B) - \mathbb{P}(A \cap B)
            \end{split}
        \end{equation*}
    \end{itemize}
\end{frame}

\section{Linear algebra}

\begin{frame}{References}
    \begin{thebibliography}{99}
        \bibitem{all_of_stats}
        Wasserman, L. (2004). \textit{All of Statistics: A Concise Course in
        Statistical Inference}. Springer Science \& Business Media.
    \end{thebibliography}
\end{frame}

\end{document}