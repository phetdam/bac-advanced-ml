%%% BAC exercise template %%%
% template modified by Derek Huang, original by Sean Cox.
\documentclass{article}
\usepackage[margin=1in]{geometry}
\usepackage{amsmath, amssymb, amsfonts, enumitem, fancyhdr, tikz}
% get rid of paragraph indent
\setlength{\parindent}{0 pt}
% allow section.equation numbering
\numberwithin{equation}{section}
% allows you to copy-paste code segments. requires pygments, which can be
% installed from PyPI with pip install Pygments.
% warning: minted does NOT work with Texmaker if you have TeX Live installed
% on WSL but Texmaker installed natively on Windows!
%\usepackage{minted}
% alternative to minted that does not require Python, LaTeX only. listings is
% however disgusting out of the box and some setup is required.
\usepackage{listings, xcolor}
% makes clickable links to sections
\usepackage{hyperref}
% make the link colors blue, as well as cite colors. urls are magenta
\hypersetup{
    colorlinks, linkcolor = blue, citecolor = blue, urlcolor = magenta
}
% fancy pagestyle so we can use fancyhdr for fancy headers/footers
\pagestyle{fancy}
% add logo in right of header. note that you will have to adjust logo path!
\fancyhead[R]{\includegraphics[scale = 0.15]{../bac_logo1.png}}
% don't show anything in the left and center header
\fancyhead[L, C]{}
% give enough space for logo by reducing top margin height, head separator,
% increasing headerheight. see Figure 1 in the fancyhdr documentation. if
% \topmargin + \headheight + \headsep = 0, original text margins unchanged.
\setlength{\topmargin}{-60 pt}
\setlength{\headheight}{50 pt}
\setlength{\headsep}{10 pt}
% remove decorative line in the fancy header
\renewcommand{\headrulewidth}{0 pt}

% color definitions for listings syntax highlighting. uses colors borrowed
% from the VS Code Dark+ and Abyss standard themes.
\definecolor{KwColor}{RGB}{153, 102, 184}     % keyword color
\definecolor{VarColor}{RGB}{86, 156, 214}     % variables/identifier color
\definecolor{StrColor}{RGB}{209, 105, 105}    % string color
\definecolor{CmtColor}{RGB}{106, 153, 85}     % comment color

% general listings configuration for all languages
\lstset{
    % change keyword, identifier, comment, string colors
    keywordstyle = \color{KwColor},
    commentstyle = \color{CmtColor},
    identifierstyle = \color{VarColor},
    stringstyle = \color{StrColor},
    % no spaces in strings
    showstringspaces = false,
    % monospace font by default
    basicstyle = \ttfamily,
    % tabsize 8 by default, this is not the 1960s
    tabsize = 4,
    % add line numbers to the left with gray typewriter font
    numbers = left,
    numberstyle = \color{gray}\ttfamily,
    % change distance from code block from 10 pt to 5 pt
    numbersep = 5 pt
}

% title, author, date
\title{Exercise 1}
\author{Derek Huang\thanks{NYU Stern 2021, BAC Advanced Team.}}
\date{February 15, 2021}

\begin{document}

\maketitle
% need to include this after making title to undo the automatic
% \thispagestyle{plain} command that is issued.
\thispagestyle{fancy}

\section{Set theory}

\subsection{Partition of $ \mathbb{R} $}

A \textit{partition} $ P $ of a non-empty set $ A $ is a set of non-empty,
disjoint sets $ A_1, A_2, \ldots \subset A $ such that
$ \bigcup_{k = 1}^\infty A_k = A $. Give a partition of $ \mathbb{R} $ into
half-open intervals whose endpoints are elements of $ 2\mathbb{Z} $, i.e. zero
or even integers.

\subsection{Slicing the unit circle}

The unit circle on $ \mathbb{R}^2 $ is the set
$ U \triangleq \{\mathbf{x} \in \mathbb{R}^2 :
\Vert\mathbf{x}\Vert_2 \le 1\} $. Define
$ \tilde{U} \triangleq U \setminus A $ as the ``sliced'' unit circle, i.e. the
subset of $ U $ where no points that lie on the $ x_1 $ or $ x_2 $ axes are
included. Define $ A $.

\subsection{Countably infinite intersection}

Suppose we have a sequence of sets $ \{A_k\}_{k \in \mathbb{N}} $ where
$ A_k \triangleq \left(-e^{-k}, 1 + e^{-k}\right) $. What is
$ \bigcap_{k = 1}^\infty A_k $?

\medskip

\textit{Remark.} $ \bigcap_{k = 1}^\infty A_k $ is \textit{closed} interval
on $ \mathbb{R} $. Try to convince yourself why this is true.

\subsection{$ \ell^p $-norm balls}

Define $ \bar{B}_{2, \ell^p}(\mathbf{x}, r) \triangleq \{\mathbf{x}' \in
\mathbb{R}^2 : \Vert\mathbf{x}' - \mathbf{x}\Vert_p \le r\} $ as the closed
$ \ell^p $-norm ball in $ \mathbb{R}^2 $ centered at $ \mathbf{x}
\in \mathbb{R}^2 $ with radius $ r \in (0, \infty) $. Recall that for
$ \mathbf{x} \triangleq [ \ x_1 \ \ldots \ x_n ]^\top \in \mathbb{R}^n $, its
$ \ell^p $-norm $ \Vert\mathbf{x}\Vert_p $ is defined such that
\begin{equation*}
    \Vert\mathbf{x}\Vert_p \triangleq
    \left(\sum_{k = 1}^n|x_k|^p\right)^{1 / p}
\end{equation*}

Consider the $ \ell^1 $-norm and the $ \ell^\infty $-norm, where
$ \Vert\mathbf{x}\Vert_\infty = \max\{x_1, \ldots x_n\} $, and define
$ B_{2, \ell^1}(\mathbf{0}, 1) $, $ B_{2, \ell^\infty}(\mathbf{0}, 1) $.
Which of these two sets includes the other and
$ B_{2, \ell^2}(\mathbf{0}, 1) $?

\medskip

\textit{Hint.} Drawing a picture might help since these norm balls are
defined on $ \mathbb{R}^2 $.

\subsection{Complex numbers}

The complex numbers $ \mathbb{C} $ are numbers of the form $ a + bi $, where
$ a, b \in \mathbb{R} $ and $ i^2 = -1 $. Prove that
$ \mathbb{R} \subset \mathbb{C} $.

\section{Probability}

\subsection{Event unions}

Let $ \Omega $ be an arbitrary sample space and let $ A, B \subseteq \Omega $
be two events. Prove using only Kolmogorov's axioms the well-known fact that
$ \mathbb{P}(A \cup B) = \mathbb{P}(A) + \mathbb{P}(B) -
\mathbb{P}(A \cap B) $.

\medskip

\textit{Hint.} Try writing $ A, B $ as unions of disjoint events.

\subsection{Sequences of coin flips}

Consider sequences of $ n \in \mathbb{N} $ independent flips of a fair coin
and note that the sample space $ \Omega \triangleq \{\text{H}, \text{T}\}^n $.
\begin{enumerate}[label = \alph*.]
    \item
    $ \forall \omega \in \Omega $, what is $ \mathbb{P}\{\omega\} $?

    \item
    $ \Omega $ is finite. What is $ |\Omega| $?

    \item
    Define $ H_n : \Omega \rightarrow \mathbb{N} \cup \{0\} $ as the random
    variable where $ \forall \omega \in \Omega $, $ H_n(\omega) $ is the
    number of heads seen during the sequence of coin flips $ \omega $. What
    is $ \mathbb{E}[H_n] $ and $ \operatorname{Var}(H_n) $?

    \medskip
    
    \textit{Hint.} Note that $ H_n \sim \operatorname{Binomial}(n, 1 / 2) $
    and that $ H_n = \sum_{k = 1}^nX_k $, where i.i.d.
    $ X_k \sim \operatorname{Bernoulli}(1 / 2) $.
\end{enumerate}

\subsection{More sequences of coin flips}

Consider sequences of $ n \in \mathbb{N} $ independent flips of an
\textit{unfair} coin, where the probability of getting heads is
$ p \in (0, 1) $. Note the sample space $ \Omega $ is the same as the
$ \Omega $ in the previous question.
\begin{enumerate}[label = \alph*.]
    \item
    Prove that when $ p \ne 1 / 2 $, $ \exists \omega_1, \omega_2 \in
    \Omega $, $ \omega_1 \ne \omega_2 $, that $ \mathbb{P}\{\omega_1\} \ne
    \mathbb{P}\{\omega_2\} $.

    \item
    Define $ U_n \triangleq \sum_{k = 1}^nX_k $, where i.i.d.
    $ X_k $ pays off $ M_1 \in (0, \infty) $ units of cash if the $ k $th
    flip is heads and pays off $ -M_2 \in (-\infty, 0) $ of cash if the
    $ k $th flip is tails. Write $ \mathbb{E}[U_n] $ as a function of $ p $.

    \item
    Suppose for a given value of $ p \in (0, 1) $ we have
    $ \mathbb{E}[U_n] = 0 $. Write $ M_1 $ as a function of $ M_2, p $.
\end{enumerate}

\subsection{The Black-Scholes underlying}

Under the Black-Scholes model, the terminal price $ S_T : \Omega \rightarrow
\mathbb{R} $ of the underlying conditional on its current price $ S_t \in
(0, \infty) $, denoted as $ S_T \mid S_t $ can be written such that for
$ Z \sim \mathcal{N}(0, 1) $, $ T > t $,
\begin{equation*}
    S_T \mid S_t \triangleq S_te^{
        \left(r - \frac{1}{2}\sigma^2\right)(T - t) + \sigma\sqrt{T - t}Z
    }
\end{equation*}
$ r, \sigma \in (0, \infty) $ are the drift and diffusion terms of the
model under $ \mathbb{Q} $, the \textit{risk-neutral probability
measure}\footnotemark\footnotetext{
    If $ \mathbb{P} $ is the real-world probability measure, one can think of
    $ \mathbb{Q} $ as the probability measure corresponding to a world where
    all prices are risk-neutral. In more technical terms, it is a probability
    measure such that the discounted price processes of all risky assets are
    local martingales. For this problem, just replace any uses of
    $ \mathbb{P} $ with $ \mathbb{Q} $.
}.
\begin{enumerate}[label = \alph*.]
    \item
    Compute $ \mathbb{E}[S_T \mid S_t] $, which gives the risk-neutral
    expectation of $ S_T $ conditional on $ S_t $.

    \medskip
    
    \textit{Hint.} You don't need to derive $ f_{S_T \mid S_t} $, the density
    of $ S_T \mid S_t $, to compute $ \mathbb{E}[S_T \mid  S_t] $.

    \item
    Let $ P_1, P_2 \in (0, \infty) $, $ P_1 < P_2 $, be two price thresholds.
    What is $ \mathbb{Q}\{S_T \in (P_1, P_2) \mid S_t\} $? Write your answer
    as the difference of two terms involving $ \Phi $, the normal cdf.

    \medskip

    \textit{Hint.} Note that $ \mathbb{Q}\{S_T \in (P_1, P_2)\} =
    \mathbb{Q}\{S_T < P_1\} - \mathbb{Q}\{S_T \le P_2\} $. Use the fact that
    inequalities are preserved under monotone and affine transforms. Your
    answer should be of the form $ \Phi(\zeta_1) - \Phi(\zeta_2) $, where
    $ \zeta_k $, $ k \in \{1, 2\} $, is a quantity involving $ S_t $, $ P_k $,
    and constants.
\end{enumerate}

\end{document}