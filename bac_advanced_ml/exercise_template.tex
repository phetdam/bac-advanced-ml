%%% BAC exercise template %%%
% template modified by Derek Huang, original by Sean Cox.
\documentclass{article}
\usepackage[margin=1in]{geometry}
\usepackage{amsmath, amssymb, amsfonts, enumitem, fancyhdr, tikz}
% get rid of paragraph indent
\setlength{\parindent}{0 pt}
% allow section.equation numbering
\numberwithin{equation}{section}
% allows you to copy-paste code segments. requires pygments, which can be
% installed from PyPI with pip install Pygments.
% warning: minted does NOT work with Texmaker if you have TeX Live installed
% on WSL but Texmaker installed natively on Windows!
%\usepackage{minted}
% alternative to minted that does not require Python, LaTeX only. listings is
% however disgusting out of the box and some setup is required.
\usepackage{listings, xcolor}
% makes clickable links to sections
\usepackage{hyperref}
% make the link colors blue, as well as cite colors. urls are magenta
\hypersetup{
    colorlinks, linkcolor = blue, citecolor = blue, urlcolor = magenta
}
% fancy pagestyle so we can use fancyhdr for fancy headers/footers
\pagestyle{fancy}
% add text in left of header
\fancyhead[R]{\includegraphics[scale = 0.15]{./bac_logo1.png}}
% give enough space for logo by reducing top margin height, head separator,
% increasing headerheight. see Figure 1 in the fancyhdr documentation. if
% \topmargin + \headheight + \headsep = 0, original text margins unchanged.
\setlength{\topmargin}{-60 pt}
\setlength{\headheight}{50 pt}
\setlength{\headsep}{10 pt}
% remove decorative line in the fancy header
\renewcommand{\headrulewidth}{0 pt}

% color definitions for listings syntax highlighting. uses colors borrowed
% from the VS Code Dark+ and Abyss standard themes.
\definecolor{KwColor}{RGB}{153, 102, 184}     % keyword color
\definecolor{VarColor}{RGB}{86, 156, 214}     % variables/identifier color
\definecolor{StrColor}{RGB}{209, 105, 105}    % string color
\definecolor{CmtColor}{RGB}{106, 153, 85}     % comment color

% general listings configuration for all languages
\lstset{
    % change keyword, identifier, comment, string colors
    keywordstyle = \color{KwColor},
    commentstyle = \color{CmtColor},
    identifierstyle = \color{VarColor},
    stringstyle = \color{StrColor},
    % no spaces in strings
    showstringspaces = false,
    % monospace font by default
    basicstyle = \ttfamily,
    % tabsize 8 by default, this is not the 1960s
    tabsize = 4,
    % add line numbers to the left with gray typewriter font
    numbers = left,
    numberstyle = \color{gray}\ttfamily,
    % change distance from code block from 10 pt to 5 pt
    numbersep = 5 pt
}

% title, author, date
\title{Exercise -1}
\author{Derek Huang}
\date{February 12, 2020}

\begin{document}

\maketitle
% need to include this after making title to undo the automatic
% \thispagestyle{plain} command that is issued.
\thispagestyle{fancy}

\section{PCA whitening}

Suppose we have centered data matrix $ \mathbf{X} \in
\mathbb{R}^{N \times d} $, where each of the $ N $ rows is a transposed
$ \mathbb{R}^d $ data sample. In principal components analysis we want to find
matrices $ \mathbf{W} \in \mathbb{R}^{d \times d},
\mathbf{Z} \in \mathbb{R}^{N \times d} $, such that
\begin{enumerate}
    \item
    $ \mathbf{W}^\top\mathbf{W} = \mathbf{WW}^\top = \mathbf{I} $, i.e.
    $ \mathbf{W} $ is an orthonormal matrix.

    \item
    The columns of $ \mathbf{W} $ are eigenvectors of the sample covariance
    matrix $ \mathbf{C}_\mathbf{X} \triangleq \frac{1}{N}\mathbf{X}^\top\mathbf{X} $
    ordered in descending order by the magnitude of their corresponding
    eigenvalue.

    \item
    $ \mathbf{X} = \mathbf{ZW}^\top $, i.e. for each data point
    $ \mathbf{x}_k $, we want $ \mathbf{W}, \mathbf{Z} $ such that
    $ \mathbf{x}_k = \mathbf{Wz}_k $, $ \mathbf{z}_k $ the $ k $th row of
    $ \mathbf{Z} $.
\end{enumerate}

Suppose we find the solution by performing [compact] singular value
decomposition on $ \mathbf{X} $. That is, we find $ \mathbf{U} \in
\mathbb{R}^{N \times d} $, $ \mathbf{\Sigma} \in \mathbb{R}^{d \times d} $,
$ \mathbf{V}^\top \in \mathbb{R}^{d \times d} $ where $ \mathbf{U} $ has
orthonormal columns, $ \mathbf{\Sigma} $ is the diagonal matrix of singular
values\footnotemark\footnotetext{
    Recall that the singular values of $ \mathbf{X} $ are the square roots of
    the eigenvalues of $ N\mathbf{C}_\mathbf{X} = \mathbf{X}^\top\mathbf{X} $.
} ordered in descending order, and $ \mathbf{V} $ is an orthonormal matrix
such that $ \mathbf{X} = \mathbf{U\Sigma V}^\top $. Recall that
$ \mathbf{X} = \mathbf{ZV}^\top $, which implies that
$ \mathbf{Z} = \mathbf{XV} $, where $ \mathbf{C}_\mathbf{Z} $ is diagonal.

\medskip

What $ \tilde{\mathbf{V}} \in \mathbb{R}^{d \times d} $ can we derive from
$ \mathbf{V} $ such that $ \tilde{\mathbf{Z}} \triangleq
\mathbf{X}\tilde{\mathbf{V}}^\top $ has
$ \mathbf{C}_{\tilde{\mathbf{Z}}} = \mathbf{I} $?

\medskip

\textit{Hint.} $ \mathbf{C}_\mathbf{Z} $ is a rescaling of a particular
diagonal matrix we already computed.

\section{Flag checking}

Consider the C function using the \href{
    https://numpy.org/devdocs/reference/c-api/array.html
}{NumPy C array API} shown on the next page. Assume it is part of a C
extension module and therefore has all necessary module initialization code
as well as the required includes.

\begin{lstlisting}[language = C]
static PyObject *
npy_sigmoid(PyObject *self, PyObject *args) {
    PyArrayObject *ar;
    if (!PyArg_ParseTuple("O!", &PyArray_Type, (PyObject **) &ar)) {
        return NULL;
    }
    if (PyArray_TYPE(ar) != NPY_DOUBLE) {
        PyErr_SetString(PyExc_TypeError, "ndarray must be of NPY_DOUBLE");
        return NULL;
    }
    double *data = (double *) PyArray_DATA(ar);
    npy_intp ar_size = PyArray_SIZE(ar);
    for (npy_intp i = 0; i < ar_size; i++) {
        data[i] = 1 / (1 + exp(-data[i]));
    }
    Py_RETURN_NONE;
}
\end{lstlisting}

Why is this code dangerous? What flags must \texttt{ar} satisfy in particular?

\end{document}